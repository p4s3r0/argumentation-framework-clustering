\chapter{Related Works}
In recent years, publications targetting the topic of clustering arguments in AFs have been released. One of the first papers been written in this field was the paper ''Existential Abstraction on Argumentation Frameworks via Clustering`` from Saribatur and Wallner \cite{DBLP:conf/kr/SaribaturW21}. Since then, a tool was created (absarg-clustering)\cite{GitHub:repoRelatedTool} to determine faithfulness and spuriousness of an AF and automatically finding non-spurious partitions. Furthermore, every two years the International Competitions on Computational Models of Argumentation (ICCMA) \cite{COMP:ICCMA2023} runs a competition, aiming to assess the state of the art in practical systems for reasoning in central argumentation formalisms.

\paragraph{Absarg-clustering} The project called \emph{absarg-clustering} \cite{GitHub:repoRelatedTool} is a tool to simplify the Dung-Style argumentation frameworks by partitioning arguments into clusters. The tool was developed in 2022 and is available on GitHub under the open-source license GPL-3.0. It is a similar implementation to our tool but uses Answer Set Programming (ASP) with the clingo solver \cite{gebser_et_al:OASIcs.ICLP.2016.2} in combination with Python3. The tool covers solutions to various problems, e.g., \ computing classical extensions, checking whether an extension is spurious, or finding spurious extensions from an abstract AF.

After some minor test-runs, the runtime of the faithful/spurious check outperforms our implementation. We speculate that this is due to the implementation of the solver. The solver is implemented so that the clusters are encoded within the rules of the logic program. This improves the runtime but reduces the user interaction. Thus, the user cannot control which arguments are concretized to generate a faithful abstract AF.

Another area for improvement of the tool is that it will not scale well to large AFs. This could be improved in future works by optimizing not for the AF argument amount but for the size of the attacks instead. This could lead, depending on the AF, to better runtimes.

Nevertheless, a fusion of the two projects could lead to a fast and interactive tool. Especially the faithful/spurious check of the absarg-clustering project would lead to a big improvement of our tool. 




\paragraph{ICCMA Competition} The International Competitions on Computational Models of Argumentation (ICCMA) runs a competition designed to foster research and development in implementing computational models of argumentation. The first competition was in 2015 and was associated with the workshop ''Theory and Applications of Formal Argument (TAFA'15)``. Back then, the covered semantics were \emph{complete}, \emph{preferred}, \emph{grounded}, and \emph{stable}. The main task was to compute semantics extensions and decide whether a given argument is credulously or skeptically inferred. Over the years, the competition evolved, and now there are multiple so-called ''tracks``, each with a different focus and different problem settings. The credulous/skeptical check is still one of the tracks called ''Dynamic Track``. After all the competitors have sent in their solutions, benchmark tests are executed for all the submissions, and a final ranking is published afterward.

Since 2015, the competition has been hosted every two years, and researchers worldwide can participate. The last competition hosted by the ICCMA in 2023 was won by the researcher team from the University Artois \& CNRS with the name \texttt{Crustabri} \cite{GitHub:repoWinnerICCMA2023} and second place was awarded to the solver from the University of Helsinki with the name \texttt{$\mu$-Toksia} \cite{DBLP:conf/kr/NiskanenJ20a}. 






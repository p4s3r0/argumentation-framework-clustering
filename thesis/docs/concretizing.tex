\chapter{Implementation}

\section{Concretizing Arguments}
Concretizing a list of arguments is done iteratively by deep copying the abstract AF $F'$ to create a new AF $F''$ and mutating it. The mutation is guided by five steps considering the unchanged abstract AF $F'$ and the concrete AF $F$. 

\paragraph{Step 1:} Each argument needing concretization is first removed from the parent cluster and added as a singleton in $F''$. 
If an argument is not part of a cluster or, in other words, is not valid, we remove it and continue with the filtered valid list.
We do not consider attacks or defenses in this step since they depend on the concrete- and abstract AFs.

\paragraph{Step 2:} We add the new attacks and defenses from all concretized arguments to the remaining clusters. We must do this after
removing the arguments from the clusters because if an argument $a$ attacks argument $b$ in the concrete AF, and $b$ is part of the cluster $F'$ in the abstract AF,
by concretizing $b$, the attack $(a,F')$ would not be valid anymore.

\paragraph{Step 3:} After adding the new attacks, we need to check which attacks from $F'$ are still valid in $F''$. If an attack is not valid
anymore through the concretization, we remove it in $F''$. An attack is not valid anymore; if we remove one of the arguments being attacked or attacked by argument $a$ from the cluster $f$ and no other attack exists, s.t. $a$ is attacked from/attacking an argument within $f$.

\paragraph{Step 4:} Selfattacks of clusters could also change by the concretization of arguments. Therefore, we need to check the clusters from 
which the arguments are concretized. If an attack within the cluster is now pulled out, we need to inspect if the cluster is still attacking itself.
If not, we will remove the self-attack.

\paragraph{Step 5:} The last step is to clean up the argumentation framework $F''$ by removing all empty clusters and mutating the clusters with exactly
one singleton to the mentioned singleton. 



